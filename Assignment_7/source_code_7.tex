\documentclass{article}
\usepackage{amsmath}
\usepackage{amsthm}
\usepackage[utf8]{inputenc}
\usepackage{graphicx}
\usepackage{subfig}
\usepackage{tikz}
\usepackage{circuitikz}
\usepackage{hhline}
\usepackage{ifthen} 
\usepackage{lscape}
\usetikzlibrary{arrows,shapes.gates.logic.US,shapes.gates.logic.IEC,calc}
\begin{document}
\title{
%\logo{
Assignment 7
%}
}
\author{Manav Garg}

\date{31 December 2020}
\maketitle


\section{Boolean Expression}
The Boolean expression of $f = A\bar{B}\bar{C}\bar{D} + \bar{A}B\bar{C}\bar{D} + AB\bar{C}\bar{D} + ABC\bar{D}$ \\ 
\section{Kmap Expression}
$f= AB\bar{D} + B\bar{C}\bar{D} + A\bar{C}\bar{D}$ \\
\\
Now, we have to make this expression in product form
So, we will use the property \overline{\overline{X}} = X\\
\\
f= AB\bar{D} + B\bar{C}\bar{D} + A\bar{C}\bar{D} \\
\\
\overline{\overline{f}} = \overline{\overline{AB\bar{D} + B\bar{C}\bar{D} + A(\bar{C}\bar{D}}} \\
\\
    \overline{\overline{f}} =\overline{(\overline{AB\bar{D}}). (\overline{B\bar{C}\bar{D}}).(\overline{A\bar{C}\bar{D}})} ....         ...by Using(\overline{X+Y+Z} = \bar{X}.\bar{Y}.\bar{D}




\subsection{COMBINATIONAL CIRCUIT}


\tikzstyle{branch}=[fill,shape=circle,minimum size=3pt,inner sep=1pt]
\begin{tikzpicture}[label distance=5mm]
\node (D) at (-1,0) {D};
\node (C) at (1,0) {C};
\node (B) at (2,0) {B};
\node (A) at (0,0) {A};
   
    \node[nand gate US, draw, logic gate inputs=nnn] at ($(nand1)+(4,-2)$) (nand1) {};
    \node[nand gate US, draw, logic gate inputs=nnn] at ($((4,-4)$) (nand2) {};
    \node[nand gate US, draw, logic gate inputs=nnn] at ($((4,-6)$) (nand3) {};
    \node[nand gate US, draw, logic gate inputs=nnnn, anchor=input 1,2,3] at ($(nand3.output)+(3,1)$) (nand4) {};
    
    
    

    \draw (D) |- (nand1.input 3);
    \draw (D) |- (nand2.input 3);
    \draw (D) |- (nand3.input 3);
    \draw (C) |- (nand1.input 2);
    \draw (C) |- (nand2.input 2);
    \draw (B) |- (nand1.input 1);
    \draw (B) |- (nand3.input 2);
    \draw (A) |- (nand2.input 1);
    \draw (A) |- (nand3.input 1);
    \draw (nand1.output) -- (nand4.input 1);
    \draw (nand2.output) -- (nand4.input 2);
    \draw (nand3.output) -- (nand4.input 3);
    \draw (nand4.output) -- node[above]{$\overline{\overline{f}} = f$} ($(nand4) + (2,0)$);
    
    \node at (nand1.bin 25) [ocirc,scale=1]
    {}, \node at (nand1.bin 40) [ocirc,scale=1]
    {}, \node at (nand2.bin 40) [ocirc,scale=1]
    {}, \node at (nand2.bin 25) [ocirc,scale=1]
    {}, \node at (nand3.bin 25) [ocirc,scale=1]
    {};
        
    }
    
    

\end{tikzpicture}


\end{document}