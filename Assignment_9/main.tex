\documentclass{article}
\usepackage[utf8]{inputenc}
\usepackage{graphicx}
\usepackage[american]{circuitikz}
\usepackage{karnaugh-map}
\usepackage{float}
\usepackage{xcolor}
\usepackage{listings}
\usepackage{enumitem}

\title{Assignment 9 EC 2014-2-13}
\author{Manav Garg}
\date{8 January 2021}

\begin{document}

\maketitle

\section{Question:}

\textbf{Find the output}

\begin{figure}[!h]
\centering
\scalebox{0.8}{
\begin{tikzpicture}

%MUX1
\draw (0,0)coordinate (O)--++(0:3)coordinate (A)--++(90:4)coordinate (B)--++(180:3)coordinate (C)--cycle;
\draw (1.5,2.4) node{4x1};
\draw (1.5,1.9) node{MUX};
\draw[->,>=stealth] ($(A)!0.79!(B)$)--++(0:3);
\draw ($(O)!0.7!(A)$)node[above]{$S_0$}--++(-90:1)node[below]{$w$};
\draw ($(O)!0.5!(A)$)node[above]{$S_1$}--++(-90:1)node[below]{$x$};
\foreach \y/\t in {0.1/0,0.2/1,0.3/2,0.4/3} {
\draw[<-,>=stealth] ($(C)! \y*2.06 !(O)$) node[right] {$I _\t$}--++(180:1);}
\draw (-2,1.95) node[left]{$w_c_c$}--(-1,1.95)--(-1,2.35)--(-1,1.54);
\draw (-1,3.175)--(-3,3.175)--(-3,-0.8)node[ground]{};
\draw (-1,0.705)--(-3,0.705);

%MUX2
\draw (6,0)coordinate (O)--++(0:3)coordinate (A)--++(90:4)coordinate (B)--++(180:3)coordinate (C)--cycle;
\draw (7.5,2.4) node{4x1};
\draw (7.5,1.9) node{MUX};
\draw[->,>=stealth] ($(A)!0.79!(B)$)--++(0:1)node[right]{$Y$};
\draw ($(O)!0.7!(A)$)node[above]{$S_0$}--++(-90:1)node[below]{$y$};
\draw ($(O)!0.5!(A)$)node[above]{$S_1$}--++(-90:1)node[below]{$z$};
\foreach \y/\t in {0.1/0,0.2/1,0.3/2,0.4/3} {
\draw ($(C)! \y*2.06 !(O)$) node[right] {$I _\t$}--++(180:0);}
\draw[->,>=stealth] (5,3.15)--(5,2.35)--(6,2.35);
\draw[<-,>=stealth] (6,1.54)--(5,1.54)--(5,-0.8);
\draw[->,>=stealth] (5,0.705)--(6,0.705);
\draw (5,-0.8)node[ground]{};

\end{tikzpicture}
}
\caption{\textit{Question figure}}
\label{mux_2}
\end{figure}

\textbf{Y=}
\begin{enumerate}[label=(\Alph*)]
\item $\bar{w}\bar{x}y + w\bar{x}y$ 
\item $\bar{w}x\bar{y} + w\bar{x}\bar{y}$
\item $\bar{w}\bar{x}\bar{y} + \bar{w}xy + wxy$
\item $none$
\end{enumerate}


\section{Solution:}


Since we have $I_0$ and $I_3$ grounded, we can take their boolean equivalents to be 0. Then, we get the following equation:

\begin{equation}
    Output = \overline{ (w + \overline{x}) (\overline{w} + x) }
\end{equation}

which can be further simplified (using de Morgan's law) to obtain:

\begin{equation}
    Output = w \overline{x} + \overline{w} x
\end{equation}

Again, the same logic can be used to obtain the result of the second MUX. Since in this case, $I_2$ and $I_3$ are grounded; hence by taking their boolean equivalents to be 0, we get the following equation from the second MUX:

\begin{equation}
    F = \overline{ \overline{(Output. \overline{y}. \overline{z})}. \overline{(Output. \overline{y}. z )}}
\end{equation}

simplifying, we get:

\begin{equation}
    F = Output.\overline{y}.\overline{z} + Output.\overline{y}.z
\end{equation}

after placing the value of $Output$ from eq.(2), and performing a few more manipulations, we get:

\begin{equation}
    F = (w\overline{x}+\overline{w}x)\overline{y}(z+\overline{z})
\end{equation}

Since $z$+$\overline{z}$=1, we finally get the desired equation:

\begin{equation}
    F = (w\bar{x}\bar{y}+\bar{w}x\bar{y})
\end{equation}
Hence, The answer of the given question is (B)


\section{Truth Table}
\begin{table}[!h]
\centering
\scalebox{1.6}{
%\resizebox{\columnwidth}{!} {
\begin{tabular}{|c|c|c|c|c|c|}
\hline
\textit{\textbf{w}} & \textit{\textbf{x}} & \textit{\textbf{y}} & \textit{\textbf{z}} & \textit{\textbf{Y}} & \textbf{Term}    \\ \hline
0                   & 0                   & 0                   & 0                   & 0                   & -                \\
0                   & 0                   & 0                   & 1                   & 0                   & -                \\
0                   & 0                   & 1                   & 0                   & 0                   & -                \\
0                   & 0                   & 1                   & 1                   & 0                   & -                \\
0                   & 1                   & 0                   & 0                   & 1                   & \textit{$\overline{w}$ x $\overline{y}$  $\overline{z}$} \\
0                   & 1                   & 0                   & 1                   & 1                   & \textit{$\overline{w}$ x $\overline{y}$ z}  \\
0                   & 1                   & 1                   & 0                   & 0                   & -                \\
0                   & 1                   & 1                   & 1                   & 0                   & -                \\
1                   & 0                   & 0                   & 0                   & 1                   & \textit{w $\overline{x}$ $\overline{y}$ $\overline{z}$} \\
1                   & 0                   & 0                   & 1                   & 1                   & \textit{w $\overline{x}$ $\overline{y}$ z}  \\
1                   & 0                   & 1                   & 0                   & 0                   & -                \\
1                   & 0                   & 1                   & 1                   & 0                   & -                \\
1                   & 1                   & 0                   & 0                   & 0                   & -                \\
1                   & 1                   & 0                   & 1                   & 0                   & -                \\
1                   & 1                   & 1                   & 0                   & 0                   & -                \\
1                   & 1                   & 1                   & 1                   & 0                   & -                \\ \hline
\end{tabular}
}
\caption{Truth Table for eq.(6)}
\label{table1}
\end{table}
\section{K-map for the function Y(w,x,y,z)}
\hspace{2cm}
\begin{figure}[h]
\centering
\begin{karnaugh-map}[4][4][1][][]
    \maxterms{0,3,4,7,8,9,10,11,12,13,14,15}
    \minterms{1,5,2,6}
    
    \implicant{1}{5}
    \implicant{2}{6}

    \draw[color=black, ultra thin] (0, 4) --
    node [pos=0.7, above right, anchor=south west] {$wx$}
    node [pos=0.7, below left, anchor=north east] {$yz$}
    ++(135:1);
\end{karnaugh-map}
\caption{K-map}
\label{kmap}
\end{figure}

The expression obtained using the K-map is the same as the one obtained earlier in eq.(6).


\end{document}