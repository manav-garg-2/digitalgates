\documentclass{article}
\usepackage[utf8]{inputenc}
\usepackage{graphicx}
\usepackage[american]{circuitikz}
\usepackage{karnaugh-map}
\usepackage{float}
\usepackage{xcolor}
\usepackage{listings}
\usepackage{enumitem}

\title{Assignment 9 EC 2014-2-13}
\author{Manav Garg}
\date{8 January 2021}

\begin{document}

\maketitle

\section{Question:}

\textbf{Find the output}

\begin{figure}[!h]
\centering
\scalebox{0.8}{
\input{mux.tex}
}
\caption{\textit{Question figure}}
\label{mux_2}
\end{figure}

\textbf{Y=}
\begin{enumerate}[label=(\Alph*)]
\item $\bar{w}\bar{x}y + w\bar{x}y$ 
\item $\bar{w}x\bar{y} + w\bar{x}\bar{y}$
\item $\bar{w}\bar{x}\bar{y} + \bar{w}xy + wxy$
\item $none$
\end{enumerate}


\section{Solution:}


Since we have $I_0$ and $I_3$ grounded, we can take their boolean equivalents to be 0. Then, we get the following equation:

\begin{equation}
    Output = \overline{ (w + \overline{x}) (\overline{w} + x) }
\end{equation}

which can be further simplified (using de Morgan's law) to obtain:

\begin{equation}
    Output = w \overline{x} + \overline{w} x
\end{equation}

Again, the same logic can be used to obtain the result of the second MUX. Since in this case, $I_2$ and $I_3$ are grounded; hence by taking their boolean equivalents to be 0, we get the following equation from the second MUX:

\begin{equation}
    F = \overline{ \overline{(Output. \overline{y}. \overline{z})}. \overline{(Output. \overline{y}. z )}}
\end{equation}

simplifying, we get:

\begin{equation}
    F = Output.\overline{y}.\overline{z} + Output.\overline{y}.z
\end{equation}

after placing the value of $Output$ from eq.(2), and performing a few more manipulations, we get:

\begin{equation}
    F = (w\overline{x}+\overline{w}x)\overline{y}(z+\overline{z})
\end{equation}

Since $z$+$\overline{z}$=1, we finally get the desired equation:

\begin{equation}
    F = (w\bar{x}\bar{y}+\bar{w}x\bar{y})
\end{equation}
Hence, The answer of the given question is (B)


\section{Truth Table}
\begin{table}[!h]
\centering
\scalebox{1.6}{
%\resizebox{\columnwidth}{!} {
\begin{tabular}{|c|c|c|c|c|c|}
\hline
\textit{\textbf{w}} & \textit{\textbf{x}} & \textit{\textbf{y}} & \textit{\textbf{z}} & \textit{\textbf{Y}} & \textbf{Term}    \\ \hline
0                   & 0                   & 0                   & 0                   & 0                   & -                \\
0                   & 0                   & 0                   & 1                   & 0                   & -                \\
0                   & 0                   & 1                   & 0                   & 0                   & -                \\
0                   & 0                   & 1                   & 1                   & 0                   & -                \\
0                   & 1                   & 0                   & 0                   & 1                   & \textit{$\overline{w}$ x $\overline{y}$  $\overline{z}$} \\
0                   & 1                   & 0                   & 1                   & 1                   & \textit{$\overline{w}$ x $\overline{y}$ z}  \\
0                   & 1                   & 1                   & 0                   & 0                   & -                \\
0                   & 1                   & 1                   & 1                   & 0                   & -                \\
1                   & 0                   & 0                   & 0                   & 1                   & \textit{w $\overline{x}$ $\overline{y}$ $\overline{z}$} \\
1                   & 0                   & 0                   & 1                   & 1                   & \textit{w $\overline{x}$ $\overline{y}$ z}  \\
1                   & 0                   & 1                   & 0                   & 0                   & -                \\
1                   & 0                   & 1                   & 1                   & 0                   & -                \\
1                   & 1                   & 0                   & 0                   & 0                   & -                \\
1                   & 1                   & 0                   & 1                   & 0                   & -                \\
1                   & 1                   & 1                   & 0                   & 0                   & -                \\
1                   & 1                   & 1                   & 1                   & 0                   & -                \\ \hline
\end{tabular}
}
\caption{Truth Table for eq.(6)}
\label{table1}
\end{table}
\section{K-map for the function Y(w,x,y,z)}
\hspace{2cm}
\begin{figure}[h]
\centering
\input{kmap.tex}
\caption{K-map}
\label{kmap}
\end{figure}

The expression obtained using the K-map is the same as the one obtained earlier in eq.(6).


\end{document}