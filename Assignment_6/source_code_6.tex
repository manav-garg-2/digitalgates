\documentclass{article}
\usepackage{amsmath}
\usepackage{amsthm}
\usepackage[utf8]{inputenc}
\usepackage{graphicx}
\usepackage{subfig}
\usepackage{tikz}
\usepackage{circuitikz}
\usepackage{hhline}
\usepackage{ifthen} 
\usepackage{lscape}
\usetikzlibrary{arrows,shapes.gates.logic.US,shapes.gates.logic.IEC,calc}
\begin{document}
\title{
%\logo{
Assignment 6 
%}
}
\author{Manav Garg}

\date{30 December 2020}
\maketitle


\section{Boolean Expression}
The Boolean expression of $f = A\bar{B}\bar{C}\bar{D} + \bar{A}B\bar{C}\bar{D} + AB\bar{C}\bar{D} + ABC\bar{D}$ \\ 
\section{Kmap Expression}
$f= AB\bar{D} + B\bar{C}\bar{D} + A\bar{C}\bar{D}$
\subsection{COMBINATIONAL CIRCUIT}


\tikzstyle{branch}=[fill,shape=circle,minimum size=3pt,inner sep=1pt]
\begin{tikzpicture}[label distance=5mm]
\node (D) at (-1,0.5) {D};
\node (C) at (1,0.5) {C};
\node (B) at (2,-2.5) {B};
\node (A) at (0.5,-3.5) {A};
    \node[not gate US, draw, rotate=-90] at ($(D)+(0,-1)$) (Not0) {};
    \node[not gate US, draw, rotate=-90] at ($(C)+(0,-1)$) (Not1) {};
    \node[and gate US, draw, logic gate inputs=nnn] at ($(and1)+(4,-2)$) (and1) {};
    \node[and gate US, draw, logic gate inputs=nnn] at ($((4,-3)$) (and2) {};
    \node[and gate US, draw, logic gate inputs=nnn] at ($((4,-4)$) (and3) {};
    \node[or gate US, draw, logic gate inputs=nnnn, anchor=input 1,2,3] at ($(and3.output)+(3,1)$) (or1) {};
    
    \foreach \i in {2,1,0}
    {\path (x\i) -- coordinate (punt\i) (x\i |- Not\i.input);
    \draw  node[] {} -| (Not\i.input);}

    \draw (Not0.output) |- (and1.input 1);
    \draw (Not0.output) |- (and2.input 1);
    \draw (Not0.output) |- (and3.input 1);
    \draw (Not1.output) |- (and1.input 2);
    \draw (Not1.output) |- (and2.input 2);
    \draw (B) |- (and1.input 3);
    \draw (B) |- (and3.input 2);
    \draw (A) |- (and2.input 3);
    \draw (A) |- (and3.input 3);
    \draw (and1.output) -- (or1.input 1);
    \draw (and2.output) -- (or1.input 2);
    \draw (and3.output) -- (or1.input 3);
    \draw (or1.output) -- node[above]{$f$} ($(or1) + (1,0)$);
    
    
        
    }
    
    

\end{tikzpicture}


\end{document}